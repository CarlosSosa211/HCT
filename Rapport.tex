\documentclass[12 pt]{article}
\usepackage[francais]{babel}
\usepackage[utf8]{inputenc}
\usepackage{float}
\usepackage{graphicx}
\usepackage{caption,subcaption}
\usepackage{listings}
\lstset{
  language=C++, 
  inputencoding=ansinew,
  basicstyle=\footnotesize,
  breaklines=true,
  keywordstyle=\bfseries,
  identifierstyle=\color{black},
  showstringspaces=false,
  numbers=left,
  numberstyle={\tiny \color{black}},
  numbersep=9pt, 
  frame=single   
}
\usepackage{booktabs}
\usepackage[margin=1in]{geometry}
\usepackage{array}
\usepackage{color} 
\usepackage{amssymb}
\usepackage{amsmath}
\definecolor{mygreen}{RGB}{28,172,0} 
\definecolor{mylilas}{RGB}{170,55,241}
\title{Interpolation par l'élément fini HCT réduit
    pour la représentation d'une surface de classe $C^1$}
\date{Janvier 2018}
\author{Carlos Sosa Marrero\\
  Damien Thomas}
\begin{document}
\pagenumbering{gobble}
\maketitle
\newpage
\pagenumbering{arabic}
\section{Structure du programme}
\subsection{Module HCT}
Le module HCT (\texttt{modHct.f90}) contient les sous-routines et fonction suivantes :
\begin{itemize}
	\item \texttt{calcBaryc(A, M, lambda, dansT)} : elle calcule les coordonnées barycentriques \texttt{lambda} du point \texttt{M} par rapport au triangle T, de sommets \texttt{A}. En plus, elle détermine si \texttt{M} appartient à T (\texttt{dansT = true}), ce qui arrive lorsque $ 0 \leq \lambda_i^T \leq 1$ pour $i = 1, 2, 3$ 
	\item \texttt{calcCoeff(foncT, p, q, u, a, b, c, d, e, g, omega)} : elle calcule les B-coefficients du triangle T en utilisant les formules fournies dans l'énoncé.
	\item \texttt{calcFoncT(fonc, n, trii, foncT)} : elle extrait dans \texttt{foncT} les valeurs de $f$ en les sommets du triangle T (regroupés dans \texttt{tri}).
	\item \texttt{calcCoordT(coord, n, trii, coordT)} : elle extrait dans \texttt{coordT} les coordonnées des sommets du triangle T (regroupés dans \texttt{trii)}.
	\item \texttt{calcCoordTi(coordT, coordOmega, i, coordTi)} : elle calcule les coordonnées des sommets du sous-triangle $T_i$, \texttt{coordTi}, à partir des coordonnées des sommets du triangle $T$, \texttt{coordT}, et de son centre de gravité, \texttt{coordOmega}.
	\item \texttt{calcGradT(derivx, derivy, n, trii, gradT)} : elle extrait dans \texttt{gradT} les valeurs du gradient de $f$ en les sommets du triangle T (regroupés dans \texttt{tri}).
	\item \texttt{calcOmega(A, Omega)} : elle calcule les coordonnés du centre de gravité, \texttt{Omega}, du triangle de sommets \texttt{A}.
	\item \texttt{calcpq(A, gradA, p, q)} : elle calcule les quantités \texttt{p} et \texttt{q}, nécessaires pour l'obtention des B-coefficients.
	\item \texttt{calcu(A, Omega, u)} : elle calcule la quantité \texttt{u}, nécessaire pour l'obtention des B-coefficients.
	\item \texttt{double precision calcS(a, b, c, d, e, g, omega, lambda, i)} : elle renvoie la valeur de $S_i$, calculée en utilisant la formule fournie dans l'énoncé.
	\item \texttt{interp(testPts, S, ntest)} : elle évalue l'interpolant, \texttt{S}, en \texttt{ntest} points de test, \texttt{testPts}.
	
	
\end{itemize}
\section{Résultats}
\end{document}
